\section{Теоретическая часть}
\subsection{Профессиональные организации по управлению проектами}

Важную роль в развитии профессионального управления проектами играют международные и национальные профессиональные ассоциации.
Деятельность профессиональных ассоциаций в области управления проектами направлена как на развитие методологии проектного менеджмента, так и на пропаганду и содействие практическому применению проектных методов управления в различных секторах экономики, государственной и социальной сферах.
Для достижения данных целей ассоциации выполняют широкий спектр функций, включая:
\begin{itemize}
	\item сбор, обобщение и распространение передового опыта в области управления проектами;
	\item научно-исследовательскую деятельность в области управления проектами;
	\item разработку стандартов, учебно-методической литературы;
	\item  определение требований к компетенции специалистов в области управления проектами и сертификацию специалистов;
	\item  определение требований к системе управления проектами в организациях, оценку и сертификацию систем управления;
	\item  проведение конференций и семинаров и многое другое.
\end{itemize}

Наиболее известными в мире ассоциациями являются Международная ассоциация управления проектами (IPMA) и Институт управления проектами США (PMI).
Большим авторитетом также пользуются национальные ассоциации управления проектами Японии (PMAJ), Великобритании (APM), Германии (GPM), Австралии (AIPM) и других стран.

\textit{International Project Management Association, IPMA}

Международная ассоциация управления проектами (IPMA) образована в 1965 г. с целью обмена опытом в области профессионального управления проектами на международном уровне и зарегистрирована в Швейцарии.
IPMA является некоммерческой организацией, и построена по принципу объединения национальных ассоциаций в области управления проектами из различных стран.
Управляющим органом IPMA является Совет делегатов стран --- участниц ассоциации.
На конец 2011 г. членами организации стали национальные ассоциации из 55 стран мира, представляющие Европу, Азию, Америку, Австралию и Африку. Количество стран, входящих в IPMA, постоянно увеличивается.

Россию, которая уже более 20 лет является членом Международной ассоциации управления проектами, в IPMA представляет национальная Ассоциация управления проектами СОВНЕТ.

Основные виды деятельности IPMA включают:
• разработку требований к компетентности специалистов в области управления проек-
тами ICB (IPMA Competence Baseline);
• разработку и поддержку по всему миру 4-уровневой системы сертификации специа-
листов в области управления проектами 4-L–C;
• разработку модели оценки проектов по качеству результатов и системы управления
IPMA Project Excellence model и проведение ежегодной процедуры оценки и выбора лучших
проектов в нескольких номинациях;
• разработку модели оценки зрелости компаний в области управления проектами
IPMA-DELTA и системы сертификации организаций;
• научные исследования и публикации в области управления проектами;
• проведение ежегодных всемирных конгрессов по управлению проектами, серий
семинаров и учебных курсов.
Ассоциация управления проектами СОВНЕТ
Ассоциация управления проектами СОВНЕТ основана в 1990 г. и представляет собой
некоммерческое партнерство, объединяющее специалистов в области управления проектами
с целью развития и продвижения профессионального управления проектами в России.
Целями Ассоциации СОВНЕТ являются:
• широкое внедрение методов и средств управления проектами в различных отраслях
экономики, видах бизнеса, социальной инфраструктуры и областях государственной и обще-
ственной жизни на территории Российской Федерации;
• развитие профессионализма и повышение качества управления проектами в России
и в мире;
• оказание организационной, методической и информационной поддержки членов
ассоциации в развитии и применении профессионального управления проектами;
• развитие и совершенствование теоретических основ и практических методов в обла-
сти управления проектами;
• расширение числа профессиональных специалистов по управлению проектами, заня-
тых в различных отраслях экономики, социальной инфраструктуре и областях обществен-
ной жизни на территории Российской Федерации;
• разработка, совершенствование, пропаганда и внедрение современных методов и
инструментальных средств управления проектами;
• организация и методическое обеспечение различных форм профессионального обу-
чения и обмена опытом, повышения квалификации, подготовки и переподготовки специа-
листов в области управления проектами;
• осуществление добровольной профессиональной сертификации организаций и спе-
циалистов по управлению проектами в соответствии с установленными национальными и
международными требованиями;
• осуществление добровольной аккредитации учебных центров и других учебных заве-
дений, осуществляющих подготовку и обучение специалистов в области управления проек-
тами в соответствии с установленными национальными и международными требованиями
к добровольной профессиональной сертификации компетентности специалистов;
• организация и проведение мероприятий, способствующих укреплению творческих
контактов и профессиональных взаимосвязей ученых и практиков в области управления
проектами на территории России и за рубежом;
• оказание практической помощи организациям в вопросах применения методологии
управления проектами, а также их партнерам, в том числе из зарубежных стран, в осуществ-
лении совместных проектов на основе профессионального управления проектами;
• содействие взаимовыгодному международному сотрудничеству с Международной
ассоциацией управления проектами IРМА, а также с другими зарубежными организациями
и компаниями, заинтересованными в развитии управления проектами.
Ассоциацией СОВНЕТ разработаны национальные требования к компетентности спе-
циалистов в области управления проектами (NCB) [НТК, 2010], которые прошли валидацию
IPMA на предмет соответствия ICB. На базе национальных требований к компетентности
осуществляется сертификация специалистов по международным стандартам.
Project Management Institute, PMI
Институт управления проектами (PMI) был основан в 1969 г. в США как некоммерче-
ская организация, объединяющая специалистов в области управления проектами. Членство
в PMI является индивидуальным, ассоциация насчитывает более 300 тыс. человек из 170
стран мира. В различных государствах и городах члены PMI объединяются в отделения для
обмена опытом и распространения знаний в области управления проектами. В России отде-
ления PMI функционируют в Москве, Санкт-Петербурге, Екатеринбурге и других городах.
Члены PMI объединяются в группы по интересам (Special Interest Groups, SIGs), дея-
тельность которых концентрируется в отдельных областях проектного менеджмента (напри-
мер, управление рисками).
Основные виды деятельности PMI:
• разработка и популяризация стандартов проектного менеджмента (широкая линейка
стандартов, основным из которых является PMBOK);
• сертификация специалистов по управлению проектами;
• оценка качества и регистрация программ обучения в области управления проектами
(Registered Education Provider, REP PMI);
• проведение ежегодных конгрессов в США, Европе, Азии;
• научные исследования и публикации в области управления проектами;
• оценка и награждение лучших проектов.
Линейка стандартов PMI по управлению проектами включает не только один из наи-
более популярных в мире стандартов The Guide to the PMBOK (Project Management Body of
Knowledge), но и стандарты по управлению программами, портфелями проектов, методоло-
гические стандарты и отраслевые адаптации стандарта по управлению проектами.
ISO/TC 258 Project, Programme and Portfolio Management
Заметную роль в области стандартизации в управлении проектами начинает играть
технический комитет, созданный Международной организацией по стандартизации – ISO.
Технический комитет ISO/TC 258 Project, Programme and Portfolio Management занима-
ется разработкой серии специализированных международных стандартов в области управ-
ления проектами, программами и портфелями проектов. Объединяет представителей наци-
ональных организаций по стандартизации из различных стран мира, в том числе из России.
В комитет входят также представители IPMA и PMI.
Одним из первых в новой серии международных стандартов по управлению проектами
ISO должен стать стандарт ISO 21500 Guidance on Project Management, выход которого был
запланирован на 2012 г.
В России при Федеральном агентстве по техническому регулированию и метрологии
(РОССТАНДАРТ) создан подкомитет по разработке стандартов в области управления про-
ектами. Подкомитет «Менеджмент проектов» входит в состав технического комитета «Стра-
тегический и инновационный менеджмент» и курирует разработку стандартов в области
управления проектами на национальном уровне. Первые стандарты в данной области вклю-
чают следующие:
• ГОСТ Р 54869-2011: Проектный менеджмент. Требования к управлению проектом.
• ГОСТ Р 54871-2011: Проектный менеджмент. Требования к управлению программой.
• ГОСТ Р 54870-2011: Проектный менеджмент. Требования к управлению портфелем
проектов.


