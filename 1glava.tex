\section{Теоретическая часть}
\subsection{Профессиональные организации по управлению проектами}

Важную роль в развитии профессионального управления проектами играют международные и национальные профессиональные ассоциации.
Деятельность профессиональных ассоциаций в области управления проектами направлена как на развитие методологии проектного менеджмента, так и на пропаганду и содействие практическому применению проектных методов управления в различных секторах экономики, государственной и социальной сферах.
Для достижения данных целей ассоциации выполняют широкий спектр функций, включая:
\begin{itemize}
	\item сбор, обобщение и распространение передового опыта в области управления проектами;
	\item научно-исследовательскую деятельность в области управления проектами;
	\item разработку стандартов, учебно-методической литературы;
	\item  определение требований к компетенции специалистов в области управления проектами и сертификацию специалистов;
	\item  определение требований к системе управления проектами в организациях, оценку и сертификацию систем управления;
	\item  проведение конференций и семинаров и многое другое.
\end{itemize}

Наиболее известными в мире ассоциациями являются Международная ассоциация управления проектами (IPMA) и Институт управления проектами США (PMI).
Большим авторитетом также пользуются национальные ассоциации управления проектами Японии (PMAJ), Великобритании (APM), Германии (GPM), Австралии (AIPM) и других стран.

\textit{International Project Management Association, IPMA}

Международная ассоциация управления проектами (IPMA) образована в 1965 г. с целью обмена опытом в области профессионального управления проектами на международном уровне и зарегистрирована в Швейцарии.
IPMA является некоммерческой организацией, и построена по принципу объединения национальных ассоциаций в области управления проектами из различных стран.
Управляющим органом IPMA является Совет делегатов стран --- участниц ассоциации.
На конец 2011 г. членами организации стали национальные ассоциации из 55 стран мира, представляющие Европу, Азию, Америку, Австралию и Африку. Количество стран, входящих в IPMA, постоянно увеличивается.

Россию, которая уже более 20 лет является членом Международной ассоциации управления проектами, в IPMA представляет национальная Ассоциация управления проектами СОВНЕТ.

Основные виды деятельности IPMA включают:
\begin{itemize}
	\item разработку требований к компетентности специалистов в области управления проектами ICB (IPMA Competence Baseline);
	\item разработку и поддержку по всему миру 4-уровневой системы сертификации специалистов в области управления проектами 4-L–C;
	\item разработку модели оценки проектов по качеству результатов и системы управления IPMA Project Excellence model и проведение ежегодной процедуры оценки и выбора лучших проектов в нескольких номинациях;
	\item разработку модели оценки зрелости компаний в области управления проектами IPMA-DELTA и системы сертификации организаций;
	\item научные исследования и публикации в области управления проектами;
	\item проведение ежегодных всемирных конгрессов по управлению проектами, серий семинаров и учебных курсов.
\end{itemize}

\textit{Ассоциация управления проектами СОВНЕТ}

Ассоциация управления проектами СОВНЕТ основана в 1990 г. и представляет собой некоммерческое партнерство, объединяющее специалистов в области управления проектами с целью развития и продвижения профессионального управления проектами в России.

Целями Ассоциации СОВНЕТ являются:
\begin{itemize}
	\item широкое внедрение методов и средств управления проектами в различных отраслях экономики, видах бизнеса, социальной инфраструктуры и областях государственной и общественной жизни на территории Российской Федерации;
	\item развитие профессионализма и повышение качества управления проектами в России и в мире;
	\item оказание организационной, методической и информационной поддержки членов ассоциации в развитии и применении профессионального управления проектами;
	\item развитие и совершенствование теоретических основ и практических методов в области управления проектами;
	\item расширение числа профессиональных специалистов по управлению проектами, занятых в различных отраслях экономики, социальной инфраструктуре и областях общественной жизни на территории Российской Федерации;
	\item разработка, совершенствование, пропаганда и внедрение современных методов и инструментальных средств управления проектами;
	\item организация и методическое обеспечение различных форм профессионального обучения и обмена опытом, повышения квалификации, подготовки и переподготовки специалистов в области управления проектами;
	\item осуществление добровольной профессиональной сертификации организаций и специалистов по управлению проектами в соответствии с установленными национальными и международными требованиями;
	\item осуществление добровольной аккредитации учебных центров и других учебных заведений, осуществляющих подготовку и обучение специалистов в области управления проектами в соответствии с установленными национальными и международными требованиями к добровольной профессиональной сертификации компетентности специалистов;
	\item организация и проведение мероприятий, способствующих укреплению творческих контактов и профессиональных взаимосвязей ученых и практиков в области управления проектами на территории России и за рубежом;
	\item оказание практической помощи организациям в вопросах применения методологии управления проектами, а также их партнерам, в том числе из зарубежных стран, в осуществлении совместных проектов на основе профессионального управления проектами;
	\item содействие взаимовыгодному международному сотрудничеству с Международной ассоциацией управления проектами IРМА, а также с другими зарубежными организациями и компаниями, заинтересованными в развитии управления проектами.
\end{itemize}

Ассоциацией СОВНЕТ разработаны национальные требования к компетентности специалистов в области управления проектами (NCB) [НТК, 2010], которые прошли валидацию IPMA на предмет соответствия ICB. На базе национальных требований к компетентности осуществляется сертификация специалистов по международным стандартам \cite[35]{aleshin}.

\textit{Project Management Institute, PMI}

Институт управления проектами (PMI) был основан в 1969 г. в США как некоммерческая организация, объединяющая специалистов в области управления проектами.
Членство в PMI является индивидуальным, ассоциация насчитывает более 300 тыс. человек из 170 стран мира.
В различных государствах и городах члены PMI объединяются в отделения для обмена опытом и распространения знаний в области управления проектами.
В России отделения PMI функционируют в Москве, Санкт-Петербурге, Екатеринбурге и других городах.

Члены PMI объединяются в группы по интересам (Special Interest Groups, SIGs), деятельность которых концентрируется в отдельных областях проектного менеджмента (например, управление рисками).

Основные виды деятельности PMI:
\begin{itemize}
	\item разработка и популяризация стандартов проектного менеджмента (широкая линейка стандартов, основным из которых является PMBOK);
	\item сертификация специалистов по управлению проектами;
	\item оценка качества и регистрация программ обучения в области управления проектами (Registered Education Provider, REP PMI);
	\item проведение ежегодных конгрессов в США, Европе, Азии;
	\item научные исследования и публикации в области управления проектами;
	\item оценка и награждение лучших проектов.
\end{itemize}

Линейка стандартов PMI по управлению проектами включает не только один из наиболее популярных в мире стандартов The Guide to the PMBOK (Project Management Body of Knowledge), но и стандарты по управлению программами, портфелями проектов, методологические стандарты и отраслевые адаптации стандарта по управлению проектами \cite[36]{aleshin}..

\textit{ISO/TC 258 Project, Programme and Portfolio Management}

Заметную роль в области стандартизации в управлении проектами начинает играть технический комитет, созданный Международной организацией по стандартизации --- ISO.

Технический комитет ISO/TC 258 Project, Programme and Portfolio Management занимается разработкой серии специализированных международных стандартов в области управления проектами, программами и портфелями проектов.
Объединяет представителей национальных организаций по стандартизации из различных стран мира, в том числе из России.
В комитет входят также представители IPMA и PMI.

Одним из первых в новой серии международных стандартов по управлению проектами ISO должен стать стандарт ISO 21500 Guidance on Project Management, выход которого был запланирован на 2012 г.

В России при Федеральном агентстве по техническому регулированию и метрологии (РОССТАНДАРТ) создан подкомитет по разработке стандартов в области управления проектами.
Подкомитет «Менеджмент проектов» входит в состав технического комитета «Стратегический и инновационный менеджмент» и курирует разработку стандартов в области управления проектами на национальном уровне \cite[37]{aleshin}.
Первые стандарты в данной области включают следующие:
\begin{itemize}
	\item ГОСТ Р 54869-2011: Проектный менеджмент. Требования к управлению проектом.
	\item ГОСТ Р 54871-2011: Проектный менеджмент. Требования к управлению программой.
	\item ГОСТ Р 54870-2011: Проектный менеджмент. Требования к управлению портфелем проектов.
\end{itemize}



