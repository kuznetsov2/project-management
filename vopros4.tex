\subsection{Процессы инициации, планирования, реализации проекта}

Инициация проекта, являясь одной из важнейших его фаз, закладывает фундамент успеха всего проекта.
Здесь определяются основные цели проекта, анализируются стратегические риски, выделяются ключевые участники проекта, определяются общие принципы организации управления проектом.

В состав процессов инициации входят:
\begin{itemize}
	\setlength\itemsep{0pt}
	\item разработка Устава проекта;
	\item анализ заинтересованных сторон ;
	\item сбор требований;
	\item стартовое совещание по проекту .
\end{itemize}

Основными задачами в ходе инициации проекта являются:
\begin{itemize}
	\setlength\itemsep{0pt}
	\item понимание основных заинтересованных сторон проекта, их интере­сов и ожиданий от проекта и его результатов;
	\item сбор требований от заказчика и иных заинтересованных сторон;
	\item формальный авторизованный старт проекта путем выпуска до­кумента <<Устав проекта>> и проведения стартового совещания по
	проекту.
\end{itemize}

Рассмотрим процессы инициации проекта более подробно.

Целью разработки Устава проекта является авторизация и формализа­ция проекта путем четкого очерчивания границ проекта, документиро­вания его целей и результатов, определения менеджера проекта, зоны его ответственности и полномочий.
Устав проекта должен связать проект со стратегическими целями организации, обосновать его необходимость, определить его содержа­ние и ответственных за реализацию.

Целью устава проекта, как исходного интеграционного документа является обеспечение однозначного понимания и фиксация:
{\setstretch{1.0}
\begin{itemize}
	\item обоснование инициации проекта;
	\item цели и результаты проекта;
	\item описание и структуру продукта проекта;
	\item ожидания ключевых участников проекта;
	\item критерии успеха проекта;
	\item фамилию менеджера проекта и зону его ответственности в проекте ;
	\item основные принципы организации проекта и управления им.
\end{itemize}
}

Устав проекта --- документ, разработка которого направлена на обеспечение следующих результатов:
\begin{itemize}
	\setlength\itemsep{0pt}
	\item авторизацию проекта;
\item определение проекта;
\item назначение менеджера проекта и распределения ролей основных участников проекта.
\end{itemize}








