\subsection{Процессы инициации, планирования, реализации проекта}

Инициация проекта, являясь одной из важнейших его фаз, закладывает фундамент успеха всего проекта.
Здесь определяются основные цели проекта, анализируются стратегические риски, выделяются ключевые участники проекта, определяются общие принципы организации управления проектом.

В состав процессов инициации входят:
\begin{itemize}
	\item [---] разработка устава проекта;
	\item [---] анализ заинтересованных сторон ;
	\item [---] сбор требований;
	\item [---] стартовое совещание по проекту .
\end{itemize}

Основными задачами в ходе инициации проекта являются:
\begin{itemize}
	\item [---]понимание основных заинтересованных сторон проекта, их интере­сов и ожиданий от проекта и его результатов;
	\item [---]сбор требований от заказчика и иных заинтересованных сторон;
	\item [---]формальный авторизованный старт проекта путем выпуска до­кумента <<Устав проекта>> и проведения стартового совещания по
	проекту.
\end{itemize}

Рассмотрим процессы инициации проекта более подробно.

Разработка устава проекта.
Целью разработки устава проекта является авторизация и формализа­ция проекта путем четкого очерчивания границ проекта, документиро­вания его целей и результатов, определения менеджера проекта, зоны его ответственности и полномочий.
Устав проекта должен связать проект со стратегическими целями организации, обосновать его необходимость, определить его содержа­ние и ответственных за реализацию.

Целью устава проекта, как исходного интеграционного документа является обеспечение однозначного понимания и фиксация:
\begin{itemize}
	\item [---] обоснование инициации проекта;
	\item [---] цели и результаты проекта;
	\item [---] описание и структуру продукта проекта;
	\item [---]ожидания ключевых участников проекта;
	\item [---]критерии успеха проекта;
	\item [---]фамилию менеджера проекта и зону его ответственности в проекте;
	\item [---]основные принципы организации проекта и управления им.
\end{itemize}

Устав проекта --- документ, разработка которого направлена на обеспечение следующих результатов:
\begin{itemize}
	\item [---]авторизацию проекта;
\item [---]определение проекта;
\item [---]назначение менеджера проекта и распределения ролей основных участников проекта.
\end{itemize}

Результатами процесса разработки устава проекта являются однознач­ное понимание содержания проекта всеми его участниками и автори­зация начала проекта лицами, принимающими решение \cite[135--137]{polkovnikov}

Анализ заинтересованных сторон.
Целью анализа заинтересованных сторон является понимание возмож­ных зон воздействия на проект со стороны его участников и внешних заинтересованных сторон путем выявления основных лиц, групп и организаций, имеющих прямые или косвенные интересы в проекте.

Заинтересованные стороны (стейкхолдеры) в проекте существуют независимо от нашего желания.
Если бы их не было, проект никогда бы не состоялся.
При этом и х интересы весьма различаются.
Одни заинтересованные стороны проект запускают, продвигают к успеху, другие стороны имеют иные интересы.

Интересы по отношению к проекту могут быть:
\begin{itemize}
	\item [---]положительными или отрицательными;
	\item [---]прямыми или косвенными ;
	\item [---]явными и неявными (скрытыми) .
\end{itemize}

Действия заинтересованных сторон иногда способствуют успеху проекта, а иногда - нет.
Заинтересованные стороны будут влиять на окружение проекта (см. рисунок \ref{fig:okuzhenie}), создавая для менеджера проекта либо позитивные условия, либо серьезные препятствия для успеха проекта.

\begin{figure}[h]
	\centering
	\includegraphics[width=\linewidth]{okruzh}
	\caption{Окружение проекта}
	\label{fig:okuzhenie}
\end{figure}

Выделяют ближнее и дальнее окружение проекта.
В ближнем окружении основными заинтересованными сторонами являются руко­водство организации, представители функциональных подразделений,
сотрудники, в дальнем --- конкуренты, органы власти, представители общественных организаций.

В ходе анализа заинтересованных сторон в проекте рекомендуется выделить основные группы заинтересованных сторон и понять их ин­тересы.
Это требуется:
\begin{itemize}
	\item [---]для назначения явных заинтересованных сторон (участников про­екта) на роли, соответствующие их интересам;
	\item [---]согласования с ключевыми внутренним и (а иногда и внешними) заинтересованными сторонами целей и результатов проекта;
	\item [---]выявления угроз и зон риска со стороны заинтересованных сторон, негативно настроенных по отношению к проекту;
	\item [---]определения потенциальных возможностей (в том числе дополни­тельных), возникающих в случае активного привлечения соответ­ствующих заинтересованных сторон;
	\item [---]установления информационных потребностей заинтересованных сторон и дальнейшего включения их в коммуникационное поле проекта.
\end{itemize}

Результатами процесса анализа заинтересованных сторон являются готовность менеджера и команды проекта к влиянию, оказываемому на проект различными заинтересованными сторонами, и минимизация негативных эффектов этого влияния.

Заинтересованные стороны и их интересы могут быть отражены в Реестре заинтересованных сторон.
Для каждой заинтересованной стороны в этом документе полезно зафиксировать:

имя человека или название группы , представляющих заинтересо­
ванную сторону ;
отношение заинтересованной стороны к проекту ( положительное ,
отрицательное , нейтральное) ;
силу возможного влияния на проект ;
степень информированности о проекте , его целях и текущем со­
стоянии;
дополнительные условия, при которых отношение к проекту может
измениться.
По итогам анализа заинтересованных сторон проекта могут быть
внесены изменения в существующие проектные документы: У став ,
План проекта и др.


