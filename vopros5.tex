\subsection{Мониторинг практического выполнения работ. Методы: простого и детального контроля, 50/50, <<по вехам>>}

Первый шаг в процессе контроля заключается в сборе и обработке данных по фактическому состоянию работ.
Руководство обязано непрерывно следить за ходом выполнения проекта, определять степень завершенности работ и, исходя из текущего состояния делать оценки параметров выполнения будущих работ.
Для этого необходимо иметь эффективные обратные связи, дающие информацию о достигнутых результатах и затратах.

Эффективным средством сбора данных являются заполненные фактическими данными и возвращенные наряды на выполнение работ или специальные отчеты, заполняемые исполнителями.

При разработке системы сбора информации менеджер проекта должен в первую очередь определить состав собираемых данных и периодичность сбора. 
Решения по данным вопросам зависят от задач анализа параметров проекта, периодичности проведения совещаний и выдачи заданий.
Детальность анализа в каждом конкретном случае определяется исходя из целей и критериев контроля проекта.
Например, если основным приоритетом является своевременность выполнения работ, то методы контроля использования ресурсов и затрат можно задействовать в ограниченном виде.

Существует два основных метода контроля фактического выполнения: «простой контроль» и «детальный контроль».

Метод простого контроля также называют методом «0--100», поскольку он отслеживает только моменты завершения детальных задач (существуют только две степени завершенности задачи: 0 и 100\%).
Другими словами, считается, что работа выполнена только тогда, кода достигнут ее конечный результат.

Метод детального контроля предусматривает выполнение оценки промежуточных состояний выполнения задачи (например, завершенность детальной задачи на 50\% означает, что, по оценкам исполнителей и руководства, цели задачи достигнуты наполовину).
Данный метод более сложен, поскольку требует от менеджера оценивать процент завершенности для работ, находящихся в процессе выполнения.

Отметим, что точное представление о состоянии выполняемых задач проекта метод «детального контроля» дает только в том случае, если оценки завершенности задач делаются корректно.
В большинстве же случаев применение метода «0--100» в сочетании с достаточной степенью детализации задач дает приемлемые результаты.

Иногда встречаются несколько модифицированные варианты метода детального контроля:

Метод 50/50 признает возможность учета некоторого промежуточного результата для незавершенных работ. Степень завершенности работы определяется в момент, когда работа израсходовала 50\% бюджета.

Метод по вехам применяется для длительных работ. Работа делится на части вехами, каждая из которых подразумевает определенную степень завершенности работы.

Используя один из этих методов, менеджер может разработать интегрированную систему контроля, которая сосредотачивает внимание на степени завершенности работ, а не только на временных и объемных параметрах проекта и удовлетворяет критериям обоснования финансирования.

Данные, необходимые для контроля основных параметров проекта, представлены в табл. \ref{control}.
\begin{table}[!h]
	\small
	\caption{Критерии для контроля и требуемые данные}
	\label{control}
	\begin{tabularx}
		\hline
		Kритерий контроля & Kоличественные данные                                                                                                                                                                                                       & Kачественные данные                                                                                 \\ \hline
		Время и стоимость & \begin{tabular}[c]{@{}l@{}}Планируемая дата начала/окончания\\ Фактическая дата начала/окончания\\ Объем выполненных работ\\ Объем предстоящих работ\\ Другие фактические затраты\\ Другие предстоящие затраты\end{tabular} &                                                                                                     \\ \hline
		Kачество          &                                                                                                                                                                                                                             & Проблемы качества                                                                                   \\ \hline
		Организация       &                                                                                                                                                                                                                             & \begin{tabular}[c]{@{}l@{}}Внешние задержки\\ Проблемы внутренней координации ресурсов\end{tabular} \\ \hline
		Содержание работ  &                                                                                                                                                                                                                             & \begin{tabular}[c]{@{}l@{}}Изменение работ\\ Технические проблемы\end{tabular}                      \\ \hline
	\end{tabularx}
\end{table}

Обычно количественные показатели собираются на уровне работ или пакетов работ и затем обобщаются для верхних уровней контроля на основании ИСР.
Поскольку оценки выполнения проекта в целом и отдельных его этапов рассчитываются на основании данных о выполнении детальных задач, важно на этапе разработки системы контроля сделать правильный выбор весовых коэффициентов формирования обобщенных оценок.

Например, использование в качестве весовых коэффициентов продолжительности задач приводит к тому, что основной вклад в процент выполнения составной задачи будут вносить наиболее длительные дочерние.
Вес задачи может устанавливаться в соответствии с ее плановой стоимостью.
Как правило, плановая стоимость является достаточно надежным показателем значимости задачи.
Иногда расходы и объемы работ не связаны напрямую, например, в случае использования в процессе реализации задач дорогих материалов и оборудования.
Возможно, более удачным в данном случае будет определять удельные веса задач на основе расходов, связанных только с использованием ресурсов, или планового объема работ. Это позволяет устранить искажения, которые стоимость основных фондов вносит в анализ расходов, связанных с оплатой ресурсов.