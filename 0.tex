\subsection*{Введение}
\addcontentsline{toc}{subsection}{Введение}

Проведение учебной практики «Управление проектами» направлено на закрепление полученных умений и навыков практической, организационной и проектной работы в условиях реального производства. 

Целью практики является получение студентом практических навыков управления проектами в соответствии с процессами, принятыми в менеджменте; техники управления проектами с использованием экономико-математических методов, проведение научно-исследовательской работы. 

При прохождении практики, исходя из поставленных целей, определяются ее задачи: 

\begin{itemize}
	\item [---] приобретение умений и навыков на основе знаний, полученных студентами в процессе теоретического обучения; 
	\item [---] овладение инновационными профессионально-практическими умениями, производственными навыками и современными методами организации выполнения научно-исследовательской работы; 
	\item [---] овладение нормами профессии в мотивационной сфере: осознание мотивов и духовных ценностей в профессии менеджера; 
	\item [---] овладение основами профессии в операционной сфере: осознание и усвоение методологии и технологии решения профессиональных задач;
	\item [---] овладение навыками организация работы исполнителей (команды исполнителей) для осуществления конкретных проектов, видов деятельности, работ; 
	\item [---] ознакомление с инновационной, в т.ч. маркетинговой и менеджерской деятельностью предприятий; 
	\item [---] овладение умениями и навыками профессиональной деятельности: технологической, технической, экономической, социальной, правовой;
	\item [---] сбор научно-информационного материала, необходимого для написания ВКР бакалавра; 
	\item [---] изучение методических подходов к принятию решений по выработке концепции проекта, его структуризации и оценке; 
	\item [---] изучение инструментария планирования и контроля хода выполнения проекта;
	\item [---] приобретение и развитие навыков экономического моделирования проектов с применением программных средств, бизнес-симуляторов;
	\item [---] создание и ведение баз данных по различным показателям функционирования организаций; 
	\item [---] оценка эффективности проектов; 
	\item [---] оценка эффективности управленческих решений; 
	\item [---] разработка бизнес-планов создания нового бизнеса; 
	\item [---] организация предпринимательской деятельности.
\end{itemize}

Место проведения учебной практики «По управлению проектами» --- компьютерный класс факультета экономики, финансов и коммерции ФГБОУ ВО Пермский ГАТУ. 

Время проведения практики «По управлению проектами»: с 09.10.2017 по 20.10.2017 г. 

Объектом исследования является Общество с ограниченной ответственностью <<Агрофирма Острожка>>.

Период исследования 2014--2016 гг.