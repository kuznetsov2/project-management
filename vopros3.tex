\subsection{Финансирование проекта}

Организация финансирования проекта подразумевает обеспечение проекта инвестиционными ресурсами, такими как денежные средства, выражаемые в денежном эквиваленте прочие инвестиции, в том числе основные и оборотные средства, имущественные права и нематериальные активы, кредиты, займы и залоги, права землепользования и пр.

Финансирование проектов --- один из видов инвестиционной деятельности, которая всегда является рисковой, особенно в нынешних социально-экономических условиях России.
Неблагополучный инвестиционный климат, законодательная база, не отвечающая требованиям мировой практики управления проектами, --- объективные причины, мешающие эффективной реализации проектов.

Финансирование проекта должно осуществляться при соблюдении таких условий:
\begin{itemize}
	\item динамика инвестиций обеспечивает реализацию проекта в соответствии с временными и финансовыми ограничениями;
	\item снижение затрат финансовых средств и рисков проекта обеспечивается за счет соответствующей структуры и источников финансирования и определенных организационных мер, в том числе налоговых льгот, гарантий, разнообразных форм участия.
\end{itemize}

Финансирование проекта включает перечисленные ниже основные стадии:
\begin{itemize}
	\item  предварительное изучение жизнеспособности проекта (определение его целесообразности по затратам и планируемой прибыли);
	\item  разработку плана реализации проекта (оценку рисков, ресурсное обеспечение и пр.);
	\item  организацию финансирования, в том числе:
	\begin{itemize}
		\item  оценку возможных форм финансирования и выбор конкретной формы,
		\item  определение финансирующих организаций,
		\item  определение структуры источников финансирования;
	\end{itemize}
	\item  контроль выполнения плана и условий финансирования.
\end{itemize}

Финансирование проектов может осуществляется следующими способами:
\begin{itemize}
	\item самофинансирование, т.е. использование в качестве источника финансирования собственных средств инвестора (из средств бюджета и внебюджетных фондов --- для государства, из собственных средств --- для предприятия);
	\item использование заемных и привлекаемых средств.
\end{itemize}

Система финансирования инвестиционных проектов включает источники и организационные формы финансирования \cite[268--269]{mazur}.

Основными организационными формами привлечения инвестиций для финансирования инвестиционных проектов в мировой практике являются:
\begin{itemize}
\item  дефицитное финансирование, означающее государственные заимствования под гарантию государства с образованием государственного долга и последующим распределением инвестиций по проектам и субъектам инвестиционной деятельности. Государство гарантирует и осуществляет возврат долга. Различают:
	\begin{itemize}
		\item государственные бюджетные кредиты на возвратной основе,
		\item ассигнования из бюджета на безвозмездной основе,
		\item финансирование по целевым федеральным инвестиционным программам,
		\item финансирование проектов из государственных международных заимствований;
	\end{itemize}
\item акционерное или корпоративное финансирование, при котором инвестируется конкретная деятельность отрасли или предприятия, в том числе:
	\begin{itemize}
		\item участие в уставном капитале предприятия,
		\item корпоративное финансирование, заключающееся в покупке ценных бумаг;
	\end{itemize}
\item проектное финансирование, при котором инвестируется непосредственно проект. Различают проектное финансирование:
	\begin{itemize}
		\item с полным регрессом на заемщика,
		\item с ограниченным правом регресса,
		\item без права регресса на заемщика.
	\end{itemize}
\end{itemize}

Проектное финансирование можно укрупненно охарактеризовать как финансирование инвестиционных проектов, при котором сам проект является способом обслуживания долговых обязательств.
Финансирующие субъекты оценивают объект инвестиций с точки зрения того, принесет ли реализуемый проект такой уровень дохода, который обеспечит погашение предоставленной инвесторами ссуды, займов или других видов капитала.

Проектное финансирование напрямую не зависит от государственных субсидий или финансовых вложений корпоративных источников.
В развитых странах этот метод используется уже десятилетия, у нас он начал применяться недавно, с выходом Закона о соглашении о разделе продукции.

Мировой рынок проектного финансирования определяется предложениями инвестиционных ресурсов, которые могут быть вложены в реализацию проектов на условиях, определенных формами и методами проектного финансирования, и спроса на эти ресурсы со стороны заказчиков, потребителей инвестиционных проектов.

Под проектным финансированием понимается предоставление финансовых ресурсов для реализации инвестиционных проектов в виде кредита без права регресса, а также с ограниченным или полным регрессом на заемщика со стороны кредитора.
Под регрессом понимается требование о возмещении предоставленной в заем суммы.
При проектном финансировании кредитор несет повышенные риски, выдавая (с точки зрения традиционных банковских кредитов) необеспеченный или не в полной мере обеспеченный кредит.
Погашение этого кредита осуществляется за счет денежных потоков, образующихся в ходе эксплуатации объекта инвестиционной деятельности.

Различают три основные формы проектного финансирования:
\begin{enumerate}
	\item [1)] финансирование с полным регрессом на заемщика, т.е. наличие определенных гарантий или требование определенной формы ограничений ответственности кредиторов проекта. Риски проекта в основном падают на заемщика, зато при этом «цена» займа относительно невысока и позволяет быстро получить финансовые средства для реализации проекта. Финансирование с полным регрессом на заемщика используется для малоприбыльных и некоммерческих проектов;
	\item [2)] финансирование без права регресса на заемщика. Кредитор при этом не имеет никаких гарантий от заемщика и принимает на себя все риски, связанные с реализацией проекта. Стоимость такой формы финансирования достаточно высока для заемщика, так как кредитор надеется получить соответствующую компенсацию за высокую степень риска. Таким образом, финансируются проекты, имеющие высокую прибыльность и дающие в результате реализации конкурентоспособную продукцию. Проекты для такой формы финансирования должны использовать прогрессивные технологии производства продукции, иметь хорошо развитые рынки, предусматривать надежные договоренности с поставщиками материально-технических ресурсов для реализации проекта и пр.;
	\item [3)] финансирование с ограниченным правом регресса. Такая форма финансирования проекта предусматривает распределение всех рисков между его участниками так, чтобы каждый участник брал на себя зависящие от него риски. В этом случае все участники принимают на себя конкретные коммерческие обязательства и цена финансирования умеренна. Участники проекта заинтересованы в его эффективной реализации, поскольку их прибыль зависит от их деятельности.
\end{enumerate}

Для российской инвестиционной практики термин «проектное финансирование» стал, с одной стороны, модным и популярным, но, с другой стороны, у нас преобладает упрощенное понимание этого термина, близкое к обычному долгосрочному кредитованию.

Следует отметить, что и на Западе нет однозначного понимания термина «проектное финансирование», это понятие используется по крайней мере в двух смыслах:
\begin{enumerate}
	\item [1)] как целевое кредитование для реализации инвестиционного проекта в любой из трех форм --- с полным регрессом, без регресса или с ограниченным регрессом кредитора на заемщика. При этом обеспечением платежных обязательств заемщика в основном являются денежные доходы от эксплуатации объекта инвестиционной деятельности, а в случае необходимости --- активы, относящиеся к инвестиционному проекту;
	\item [2)] как способ консолидации различных источников финансирования и комплексного использования разных методов финансирования конкретных инвестиционных проектов и оптимального распределения связанных с реализацией проектов финансовых рисков.
\end{enumerate}

В настоящее время преобладает второе понимание проектного финансирования.

В целом проектное финансирование имеет историю, составляющую около четверти века.
В 1970-е гг. развитие инвестиций в нефтегазовую промышленность, обеспечивающее прибыльность в сотни и тысячи процентов в год, заставило банки перейти от пассивной роли кредиторов (когда потенциальные заемщики идут в банк и просят денег) к активному поиску форм и методов кредитования высокоприбыльных инвестиционных проектов, в первую очередь в нефтяном и газовом секторе экономики.
Банки брали на себя повышенные риски и кредитовали заемщиков на условиях проектного финансирования «без регресса» или «с ограниченным регрессом».

В отличие от традиционных форм кредитования проектное финансирование позволяет:
\begin{itemize}
	\item более достоверно оценить платежеспособность и надежность заемщика;
	\item рассмотреть весь инвестиционный проект с точки зрения жизнеспособности, эффективности, реализуемости, обеспеченности, рисков;
	\item прогнозировать результат реализации инвестиционного проекта.
\end{itemize}

Говоря о трудностях применения проектного финансирования в России, следует отметить, что в промышленно развитых странах сегодня в расчеты финансово-коммерческой эффективности проектов закладывается возможное отклонение основных показателей в худшую сторону в размере 5--10\%, а в наших условиях необходимы «допуски» не менее 20--30\%.
А это дополнительные затраты, связанные с резервированием средств для покрытия непредвиденных издержек.
Тем не менее проектному финансированию нет альтернативы.

Проектное финансирование открыло новые направления на рынке банковских услуг.
Банки выступают при этом в разных качествах:
\begin{itemize}
	\item  как банки-кредиторы;
	\item  гаранты;
	\item  инвестиционные брокеры (инвестиционные банки);
	\item  финансовые консультанты;
	\item  инициаторы создания и/или менеджеры банковских консорциумов;
	\item  институциональные инвесторы, приобретающие ценные бумаги проектных компаний;
	\item  лизинговые организации и т.д.
\end{itemize}

Важным и новым видом деятельности на рынке проектного финансирования является консалтинг в этой области, осуществляемый специализированными банками-консультантами по следующему набору услуг:
\begin{itemize}
	\item поиск, отбор и оценка инвестиционных проектов;
	\item подготовка технико-экономических обоснований;
	\item разработка схем финансирования, ведение предварительных переговоров с банками, фондами и другими институтами на предмет их совокупного участия в финансировании;
	\item подготовка всего пакета документов по проекту;
	\item оказание содействия в ведении переговоров и подписании кредитных соглашений, а также соглашений о создании консорциумов и т.д.
\end{itemize}

Банки-консультанты подготавливают комплект документов по проекту чаще всего по специальному заказу коммерческих банков или промышленных компаний.
В некоторых странах банк-консультант имеет право и сам участвовать в финансировании проекта, доказывая тем самым объективность своих оценок и серьезность своих рекомендаций.
Но, например, в Великобритании существует разделение консалтинговых и финансирующих функций и банк-консультант не участвует в финансировании.

Проектное финансирование имеет для заемщика средств ряд преимуществ, и прежде всего ограничение ответственности перед кредитором, но также и определенные недостатки:
\begin{itemize}
	\item высокие предварительные затраты потенциального заемщика для разработки детальной заявки банку на финансирование проекта на прединвестиционной фазе (по подготовке технико-экономического обоснования, на уточнение запасов полезных ископаемых, экологическую оценку воздействия будущего проекта на окружающую среду, обширные маркетинговые исследования и другие вспомогательные предпроектные работы и исследования);
	\item сравнительно долгий период до принятия решения о финансировании, что связано с тщательной оценкой предпроектной документации банком и большим объемом работ по организации финансирования (создание банковского консорциума и т.д.);
	\item повышение процента по кредиту в связи с высокими рисками, а также рост расходов на оценку проекта, организацию финансирования, надзор и т.д.;
	\item более жесткий, чем при традиционном банковском кредитовании, контроль со стороны банка (банковского консорциума) по всем аспектам деятельности заемщика;
	\item определенную потерю заемщиком независимости, если кредитор оговаривает за собой право приобретения акций компании в случае удачной реализации проекта.
\end{itemize}

Проектное финансирование не всегда целесообразно, иногда для заемщика предпочтительнее традиционные схемы финансирования инвестиционных проектов, такие как кредиты под залоговое обеспечение, гарантии и поручительства, эмиссия акций и облигаций, лизинг и т.д.

