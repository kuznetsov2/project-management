\subsection{Финансирование проекта}

Организация финансирования проекта подразумевает обеспечение проекта инвестиционными ресурсами, такими как денежные средства, выражаемые в денежном эквиваленте прочие инвестиции, в том числе основные и оборотные средства, имущественные права и нематериальные активы, кредиты, займы и залоги, права землепользования и пр.

Финансирование проектов --- один из видов инвестиционной деятельности, которая всегда является рисковой, особенно в нынешних социально-экономических условиях России.
Неблагополучный инвестиционный климат, законодательная база, не отвечающая требованиям мировой практики управления проектами, --- объективные причины, мешающие эффективной реализации проектов.

Финансирование проекта должно осуществляться при соблюдении таких условий:
\begin{itemize}
	\item динамика инвестиций обеспечивает реализацию проекта в соответствии с временными и финансовыми ограничениями;
	\item снижение затрат финансовых средств и рисков проекта обеспечивается за счет соответствующей структуры и источников финансирования и определенных организационных мер, в том числе налоговых льгот, гарантий, разнообразных форм участия.
\end{itemize}

Финансирование проекта включает перечисленные ниже основные стадии:
\begin{itemize}
	\item  предварительное изучение жизнеспособности проекта (определение его целесообразности по затратам и планируемой прибыли);
	\item  разработку плана реализации проекта (оценку рисков, ресурсное обеспечение и пр.);
	\item  организацию финансирования, в том числе:
	\begin{itemize}
		\item  оценку возможных форм финансирования и выбор конкретной формы,
		\item  определение финансирующих организаций,
		\item  определение структуры источников финансирования;
	\end{itemize}
	\item  контроль выполнения плана и условий финансирования.
\end{itemize}

Финансирование проектов может осуществляется следующими
способами:
• самофинансирование, т.е. использование в качестве источника
финансирования собственных средств инвестора (из средств
бюджета и внебюджетных фондов — для государства, из соб
ственных средств — для предприятия);
• использование заемных и привлекаемых средств.
Система финансирования инвестиционных проектов включает ис
точники и организационные формы финансирования.








