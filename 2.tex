\subsection{Анализ финансово-хозяйственной деятельности ООО «Агрофирма Острожка»}
\subsubsection*{Структура имущества и источники его формирования}
% Please add the following required packages to your document preamble:
% \usepackage{multirow}
\begin{table}[]
	\small
	\centering
	\caption{Анализ бухгалтерского баланса}
	\label{aktiv}
	\setlength{\extrarowheight}{1.2mm}
	\begin{tabularx}{\textwidth}{|p{4.52cm}|K{1.3cm}|K{1.3cm}|K{1.3cm}|K{1.25cm}|K{1.25cm}|K{1.25cm}|K{1.25cm}|}
		\hline
		\multirow{2}{3cm}{Наименование показателя}                              & \multirow{2}{2cm}{На 31 декабря 2016 г.} & \multirow{2}{2cm}{На 31 декабря 2015 г.} & \multirow{2}{2cm}{На 31 декабря 2014 г.} & \multicolumn{2}{c|}{\multirow{1}{3cm}{Изменения за анализируемый период}} & \multicolumn{2}{c|}{Удельный вес}    \\ [5ex] \cline{5-8} 
		&                                        &                                        &                                        & Тыс. руб.                   & \%                       & На начало периода & На конец периода \\ \hline
		\multicolumn{8}{|c|}{АКТИВ} \\ 
		\multicolumn{8}{|c|}{I. ВНЕОБОРОТНЫЕ АКТИВЫ} \\ \hline
		Основные средства & 67181 & 49361 & 44533 & 22648 & 50,86 & 50,94 & 59,45            \\ \hline
		Прочие внеоборотные активы & - & 341 & 341 & -341 & -100 & 0,39 & 0                \\ \hline
		Итого по разделу I                                                    & 67181                                  & 49702                                  & 44874                                  & 22301                       & 49,71                    & 51,33             & 59,45            \\ \hline
		\multicolumn{8}{|c|}{II. ОБОРОТНЫЕ АКТИВЫ}\\ \hline
		Запасы                                                                & 40324                                  & 41052                                  & 40649                                  & -325                        & -0,80                    & 46,50             & 35,68            \\ \hline
		Дебиторская задолженность                                             & 5323                                   & 6505                                   & 1223                                   & 4100                        & 335,24                   & 1,40              & 4,71             \\ \hline
		Денежные средства и денежные эквиваленты                              & 174                                    & 140                                    & 676                                    & -502                        & -74,26                   & 0,77              & 0,15             \\ \hline
		Итого по разделу II                                                   & 45821                                  & 47697                                  & 42548                                  & 3273                        & 7,69                     & 48,67             & 40,55            \\ \hline
		БАЛАНС                                                                & 113002                                 & 97399                                  & 87422                                  & 25580                       & 29,26                    & 100               & 100              \\ \hline
		\multicolumn{8}{|c|}{ПАССИВ}\\
		\multicolumn{8}{|c|}{III. КАПИТАЛ И РЕЗЕРВЫ}\\ \hline
		Уставный капитал (складочный капитал, уставный фонд, вклады товарищей) & 700 & 700 & 700 & 0 & 0 & 0,80 & 0,62 \\ \hline
		Добавочный капитал &27063 &27063 &27063 &0 &0 &30,96 &23,95 \\ \hline
		Резервный капитал& 101& 101&101 &0 &0 &0,12 &0,09 \\ \hline
		Нераспределенная прибыль (непокрытый убыток)& 39340& 31424& 24780& 14560& 58,76& 28,35& 34,81\\ \hline
		Итого по разделу III& 67204& 59288& 52644&14560 &27,66 &60,22 &59,47 \\ \hline
		\multicolumn{8}{|c|}{IV. ДОЛГОСРОЧНЫЕ ОБЯЗАТЕЛЬСТВА}\\ \hline
		Заемные средства&9951 &14600 &20022 &-10071 &-50,30 &22,90 &8,81 \\ \hline
		Прочие обязательства&16808 &5032 &- &16808 &- &0 &14,87 \\ \hline
		Итого по разделу IV& 26759& 19632& 20022& 6737& 33,65& 22,90& 23,68\\ \hline
		\multicolumn{8}{|c|}{V. КРАТКОСРОЧНЫЕ ОБЯЗАТЕЛЬСТВА}\\ \hline
		Заемные средства& -& 2000& -& 0& -& 0&0 \\ \hline
		Кредиторская задолженность&19039 &16479 &14756 &4283 &29,03 &16,88 &16,85 \\ \hline
		Итого по разделу V&19039 &18479 &14756 &4283 &29,03 &16,88 &16,85 \\ \hline
		БАЛАНС& 113002&97399 &87422 &25580 &29,26 &100 &100 \\ \hline
	\end{tabularx}
\end{table}

Для выполнения детального анализа структуры имущества организации и источников его формирования в разрезе разделов и статей баланса рассмотрим данные. представленные в таблице \ref{aktiv}


Вертикальный и горизонтальный анализы бухгалтерской отчетности показывают следующее соотношение: 59,45\% и 40,55\% составляют внеоборотные и оборотные активы соответственно. Активы организации в течение анализируемого периода выросли на 25580 тыс. руб. или 29,26\%, что довольно существенно. Рост величины активов организации выражен, в основном, ростом объема основных средств и, в меньшей степени, ростом дебиторской задолженности. 

Из пассивной части бухгалтерского баланса видно, что на 14560 тыс. руб. или 58,76\% выросла нераспределенная прибыль организации. Также увеличение объема основных средств обусловлено увеличением таких статей как прочие долгосрочные обязательства и кредиторская задолженность.

В целом, в исследуемом периоде ООО «Агрофирма Острожка»  было зависимо от заемных источников финансирования.

Как видим, Обществу средств собственного капитала, краткосрочных и долгосрочных кредитов и займов недостаточно для финансирования материальных оборотных средств.

Таким образом, в настоящий момент финансирование запасов и затрат производится за счет средств, образующихся в результате замедления погашения кредиторской задолженности.

\subsubsection*{Анализ эффективности деятельности предприятия}

В таблице \ref{pokazat} представлены основные финансовые результаты деятельности ООО «Агрофирма Острожка» 
\begin{table}[!hb]
	\small
	\centering
	\caption{Экономические показатели деятельности предприятия ООО «Агрофирма Острожка», тыс. руб.}
	\label{pokazat}
	\setlength{\extrarowheight}{1.2mm}
	\begin{tabularx}{\textwidth}{|p{4.3cm}|K{1cm}|K{1cm}|K{1cm}|K{1.5cm}|K{1.5cm}|K{1.6cm}|K{1.6cm}|}
		\hline
		&&&&\multicolumn{2}{c|}{\multirow{1}{3cm}{Абсолютное изменение}} & \multicolumn{2}{c|}{\multirow{1}{3cm}{Темп роста (снижения),\%}} \\
		\multirow{2}{*}{Наименование}       & \multirow{2}{*}{2014 г.} & \multirow{2}{*}{2015 г.} & \multirow{2}{*}{2016 г.}&\multicolumn{2}{c|}{}&\multicolumn{2}{c|}{} \\ \cline{5-8} 
		&                           &                           &                           & 2015 г. к 2014 г.       & 2016 г. к 2015 г.       & 2015 г. к 2014 г.         & 2016 г. к 2015 г.         \\ \hline
		Выручка                             & 68683                     & 85154                     & 77717                     & 16471               & (7437)               & 123,98                & 91,27                 \\ \hline
		Себестоимость продаж                & 67062                     & 84929                     & 73110                     & 17867               & (11819)              & 126,64                & 86,08                 \\ \hline
		Валовая прибыль (убыток)            & 1621                      & 225                       & 4607                      & (1396)               & 4382                & 13,88                 & 2047,56              \\ \hline
		Прибыль (убыток) от продаж          & 1621                      & 225                       & 4607                      & (1396)               & 4382                & 13,88                 & 2047,56              \\ \hline
		Проценты к уплате                   & 2739                      & 2383                      & 1846                      & (356)                & (537)                & 87,00                 & 77,47                 \\ \hline
		Прочие доходы                       & 10533                     & 12558                     & 8002                      & 2025                & (4556)               & 119,23                & 63,72                 \\ \hline
		Прочие расходы                      & 6948                      & 3733                      & 2659                      & (3215)               & (1074)               & 53,73                 & 71,23                 \\ \hline
		Прибыль (убыток) до налогообложения & 2467                      & 6667                      & 8104                      & 4200                & 1437                & 270,25              & 121,55                \\ \hline
		Чистая прибыль (убыток)             & 2467                      & 6644                      & 7916                      & 4177                & 1272                & 269,32              & 119,15                \\ \hline
		Рентабельность продаж, \%           & 3,6                       & 7,8                       & 10,2                      & 4,2                 & 2,4                 & Х                     & Х                     \\ \hline
	\end{tabularx}
\end{table}

Проанализировав таблицу 2, можно сделать вывод, что выручка в 2015 году по сравнению к 2014 году увеличилась на 16471 тыс. руб. или на 23,9\%, но 2016
году по сравнению к 2015 году снизилась на 7437 тыс. руб., то есть на 9\% от уровня предшествующего года. Основная причина — снижение объема реализованной продукции.

В 2015 году происходит увеличение себестоимости продаж — темп роста составил 126,64\% в связи ростом цен на покупные корма и семена. В 2016 году себестоимость продаж снизилась на 86,08\% — это обусловлено в первую очередь снижением объемов продаж и выручки о реализации продукции за 2016 год.

Валовая прибыль за анализируемый период имеет нестабильную динамику. Стоит отметить, что положительным фактом является ее значительный рост в 2016 году 4607 тыс. руб. Это свидетельствует о повышении эффективности управления ресурсами предприятия, что подтверждается показателями чистой прибыли и рентабельности продаж, который в 2016 году достиг показателя в 10,2\%. Таким образом, можно сделать вывод, что Общество динамично развивается.

\subsubsection*{Коэффициентный анализ}

\textit{Коэффициент абсолютной ликвидности} показывает, какую часть текущей краткосрочной задолженности предприятие может погасить в ближайшее время за счет денежных средств и приравненных к ним финансовых вложений. Норматив данного коэффициента больше 0,2--0,5.
\[ \text{Л} = \dfrac{( \text{ден. ср-ва} + \text{фин. вложения} )}{\text{текущие обязательства}} \]

\textit{Рентабельность продаж по прибыли от продаж} показывает, сколько прибыли приходится на единицу реализованной продукции. Рост данного показателя является следствием роста цен при постоянных затратах на производство реализованной продукции или снижения затрат  на производство при постоянных ценах.
\[ \text{Р}_{\text{п}} = \dfrac{\text{прибыль от продаж}}{\text{выручка}} \times 100\% \]

\textit{Рентабельность продаж по чистой прибыли} показывает эффективность использования всего имущества предприятия. Снижение показателя свидетельствует о падающем спросе на продукцию предприятия и о перенакопления активов.
\[ \text{Р}_{\text{пч}} = \dfrac{\text{чистая прибыль}}{\text{выручка}} \times 100\% \]

\textit{Коэффициент общей оборачиваемости капитала} (ресурсоотдача) показывает эффективность использования имущества. Отражает скорость оборота.
\[ \text{Об.к.} = \dfrac{\text{выручка}}{\text{среднегод. стоимость активов}}
\]

\textit{Коэффициент оборачиваемости оборотных средств} показывает скорость оборота всех оборотных средств предприятия.
\[ \text{Об.об.с.} = \dfrac{\text{выручка}}{\text{среднегод. стоимость оборотных активов}} \]

\textit{Период погашения дебиторской задолженности} (в днях) показывает,  за сколько в среднем дней погашается дебиторская задолженность предприятия.
\[ \text{Д} = \dfrac{\text{среднегод. стоимость деб. задолженности}}{\text{выручка}}
\]

Расчет данных показателей приведен в таблице \ref{koef}.
			
\begin{table}[!ht]
	\small
	\centering
	\caption{Расчет коэффициентов}
	\label{koef}
	\setlength{\extrarowheight}{1.2mm}
	\begin{tabularx}{\textwidth}{|p{9.55cm}|K{1.9cm}|K{1.9cm}|K{1.9cm}|}
		\hline
		Название коэффициента & 2014 г. & 2015 г. & 2016 г. \\ \hline
		Коэффициент абсолютной ликвидности             & 0,05 & 0,01 & 0,01 \\ \hline
		Рентабельность продаж по прибыли от продаж, \% & 2,36 & 0,26 & 5,93 \\ \hline
		Рентабельность продаж по чистой прибыли, \%    & 3,59 & 7,80 & 10,19 \\ \hline
		Коэффициент общей оборачиваемости капитала     & 0,79 & 0,92 & 0,74 \\ \hline
		Коэффициент оборачиваемости оборотных средств  & 1,61 & 1,89 & 1,66 \\ \hline
		Период погашения дебиторской задолженности     & 0,02 & 0,05 & 0,08 \\ \hline
	\end{tabularx}
\end{table}

В динамике трех лет коэффициент абсолютной ликвидности уменьшается и не соответствует нижней границе нормы.

Рентабельность продаж показывает нестабильную динамику, но в целом имеет тенденцию к увеличению, что говорит об увеличении эффективности предприятия.

Коэффициент общей оборачиваемости капитала в 2016 г. снизился за счет увеличения доли основных средств.

Коэффициент оборачиваемости оборотных средств также нестабилен, что показывает увеличение потребности в оборотных средствах.

Период погашения дебиторской задолженности в 2016 г. продолжает увеличиваться, следовательно увеличивается период времени получения денежных средств от дебиторов.










\subsubsection*{Анализ финансовой устойчивости}

\begin{table}[!hb]
	\small
	\centering
	\caption{Показатели финансовой устойчивости ООО «Агрофирма Острожка», за 2014--2016 гг. (тыс. руб.)}
	\label{finust}
	\setlength{\extrarowheight}{1.2mm}
	\begin{tabularx}{\textwidth}{|p{4.7cm}|K{1.7cm}|K{1.7cm}|K{1.7cm}|K{2.2cm}|K{2.2cm}|}
		\hline
		\multirow{2}{*}{Финансовые показатели}                                                         & \multirow{2}{2cm}{На 31.12.2014} & \multirow{2}{2cm}{На 31.12.2015} & \multirow{2}{2cm}{На 31.12.2016} & \multicolumn{2}{c|}{Темп роста (снижения), \%} \\ \cline{5-6} 
		& & & & 2016 г. к 2014 г. & 2016 г. к 2015 г. \\ \hline
		Капитал и резервы                                                                              & 52644                          & 59288                          & 67204                          & 127,66                & 113,35                \\ \hline
		Внеоборотные активы                                                                            & 44874                          & 49702                          & 67181                          & 149,71                & 135,17                \\ \hline
		Наличие собственных оборотных средств (СОС)                                                    & 7770                           & 9586                           & 23                             & 0,30                  & 0,24                  \\ \hline
		Долгосрочные обязательства                                                                     & 20022                          & 19632                          & 26759                          & 133,65                & 136,30                \\ \hline
		Наличие собственных и долгосрочных заемных источников формирования запасов                     & 27792                          & 29218                          & 26782                          & 96,37                 & 91,66                 \\ \hline
		Заемные средства                                                                               & 0                              & 2000                           & 0                              & 0                     & 0                     \\ \hline
		Общая величина источников формирования запасов                                                 & 55584                          & 58436                          & 53564                          & 96,37                 & 91,66                 \\ \hline
		Общая величина запасов                                                                         & 40649                          & 41052                          & 40324                          & 99,20                 & 98,23                 \\ \hline
		Излишек (+), недостаток (-) с.о.с.                                                             & -32879                         & -31466                         & -40301                         & 122,57                & 128,08                \\ \hline
		Излишек (+), недостаток (-) собственных и долгосрочных заемных источников формирования запасов & -12857                         & -11834                         & -13542                         & 105,33                & 114,43                \\ \hline
		Излишек (+), недостаток (-) общей величины основных источников формирования запасов            & 14935                          & 17384                          & 13240                          & 88,65                 & 76,16                 \\ \hline
		Трехкомпонентный показатель финансовой  устойчивости                                           & 001                            & 001                            & 001                            & Х                     & Х                     \\ \hline
	\end{tabularx}
\end{table}

Финансовая устойчивость (независимость) --- это способность организации продолжать основную деятельность при единовременном погашении всех заёмных средств. Существует три показателя,  с помощью которых можно охарактеризовать финансовую устойчивость предприятия:
\begin{enumerate}
	\item [---] излишек (недостаток) собственных оборотных средств;
	\item [---] излишек (недостаток) собственных и долгосрочных заёмных источников формирования запасов;
	\item [---] излишек (недостаток) общей величины основных источников для формирования запасов.
\end{enumerate}

С помощью этих показателей можем определить трёхкомпонентный показатель типа финансовой ситуации.

Типы финансовой ситуации:
\begin{enumerate}
\item [---] абсолютная устойчивость (1,1,1);
\item [---] нормальная устойчивость (0,1,1);
\item [---] неустойчивое финансовое положение (0,0,1);
\item [---] кризисное финансовое положение (0,0,0).
\end{enumerate}

В таблице \ref{finust} отражена финансовая устойчивость предприятия.

% Please add the following required packages to your document preamble:
% \usepackage{multirow}


В исследуемом периоде ООО «Агрофирма Острожка»  было зависимо от заемных источников финансирования.

Как видим, Обществу средств собственного капитала, краткосрочных и долгосрочных кредитов и займов недостаточно для финансирования материальных оборотных средств.

Таким образом, в настоящий момент финансирование запасов и затрат производится за счет средств, образующихся в результате замедления погашения кредиторской задолженности.

Трехкомпонентный показатель финансовой устойчивости на протяжении трех последних лет остается неизменным и показывает на неустойчивое финансовое состояние, которое сопряжено нарушением платежеспособности, но при этом сохраняется возможность восстановления равновесия путем пополнения источников собственных средств за счет сокращения дебиторской задолженности и ускорения оборачиваемости запасов.








